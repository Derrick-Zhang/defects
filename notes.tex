


\documentclass[12pt]{article}
\usepackage{amssymb}
\usepackage{amsfonts}
\usepackage{youngtab}
\usepackage{amsmath,amssymb,latexsym,cite}
\usepackage{amsthm}
\usepackage{cite}
%\usepackage[a-1b]{pdfx}    % archival
\usepackage[]{hyperref}
\usepackage{xcolor}
\oddsidemargin 0in \textwidth 6.5in \topmargin 0in \headheight 0in
\textheight 8.5in
\parskip 2ex
\input xy
\xyoption{all}

                                              
\begin{document}

\vspace*{0.5in}

\begin{center}

{\large\bf Notes on Schubert cycles and defects}

\end{center}


\section{Ordinary Grassmannians}

**** Cyril's paper:  \cite{Closset:2023bdr}


Consider $G(k,N)$.  Consider the Schubert cycle corresponding to a single
box.

From (3.63), $r_1 = 1$, $M_1 = k$, and $\varphi_1^2$ is a bifundamental
$U(1) \rightarrow U(k)$.  We have $J$ terms
\begin{equation}
\varphi_1^2 \phi^{(1)}_{\alpha} \Lambda^{(1)}_{\alpha},
\end{equation}
where $\alpha \in \{1 \cdots M_1\}$.
$\phi$ is the $k \times N$ matrix defining the original Grassmannian.
It's broken into a $k \times M_1$ submatrix $\phi^{(1)}$,
and a $k \times (N-k)$ submatrix $\phi^{(0)}$ (which does not appear above).

See p 28 of Cyril et al.

So, the $J$ terms in this case are a product of $k \times 1$, $k \times k$
and $1 \times k$ matrices, and it looks like their effect is to give a mass
to a $1 \times k$ submatrix of $\phi^{(1)}$, which then has vanishing vev,
as appropriate for one of the Schubert cycles.

Note the flavor symmetry $U(N)$ is broken to $U(k) \times U(N-k)$.



\subsection{Basics of Schubert varieties}

**** Early in \cite{fp} are some good basics on Schubert cells.
See p 18 for some exact sequences; p 19 for a description of Schubert
cells in terms of vanishing matrix entries, ala Cyril.


**** From \cite[chapter 1.5]{gh}:

First, recall that we can give a cell decomposition of ${\mathbb P}^n$,
writing it as a union 
\begin{equation}
{\mathbb P}^n \: = \: {\mathbb C}^n \cup {\mathbb C}^{n-1} \cup
\cdots \cup {\mathbb C}^1 \cup {\mathbb C}^0,
\end{equation}
by first choosing a flag
\begin{equation}
V \: = \: \{ V_1 \subsetneq V_2 \subsetneq \cdots \subsetneq V_n 
\subsetneq {\mathbb C}^{n+1} \}
\end{equation}
and take 
\begin{equation}
W_i \: = \: {\mathbb C}^{i-1} \: = \: \{
\ell \in {\mathbb C}^{n+1} \, | \,
\ell \in V_i, \: \ell \not\in V_{i=1} \}.
\end{equation}
The $W_i$ then form the cell decomposition of the projective space.

***** RESTART NOTATION:  $n$ instead of $N$ below. ****

**** Following the discussion of \cite{gh} *very* closely,

we can do the same thing in Grassmannians.
Consider $G(k,n)$.  Let $\Lambda \in G(k,n)$ be a
Let $\lambda_1, \cdots, \lambda_k$ be a nonincreasing set of integers
$\leq n-k$.  
Fix a flag consisting of vector spaces $V_i \subsetneq V_{i+1}$ for all
$i$.  The Schubert cell is the set of $\Lambda \in G(k,n)$ whose intersection
with the $V_i$ is of a specific dimension.
Specifically, the Schubert cells corresponding to the $\lambda_i$ is
\begin{equation}
W_{\lambda_1, \cdots, \lambda_k} \: = \: \{ \Lambda \in G(k,n) \, | \,
\dim \left( \Lambda \cap V_{n-k+i-\lambda_i} \right) = i \}.
\end{equation}
The closure is given by
\begin{equation}
\overline{W}_{\lambda_1, \cdots, \lambda_k} \: = \: \{ \Lambda \in G(k,n) \, | \,
\dim \left( \Lambda \cap V_{n-k+i-\lambda_i} \right) \geq  i \}.
\end{equation}


**** Construct some examples and compare to Cyril.

***** GH p 196 has a big matrix illustrating this in some generality.


Now, let's compare to the construction of \cite[section 3.3.1]{Closset:2023bdr},
in the Grassmannian $G(k,n)$.
For a given Young diagram $\lambda$, define integers 
$\{ \alpha_1, \cdots, \alpha_k \}$,
\begin{equation}
1 \leq \alpha_1 < \alpha_2 < \alpha_3 < \cdots < \alpha_k \leq n,
\end{equation}
by
\begin{equation}
\lambda \: = \: [ \alpha_k - k, \cdots, \alpha_1 - 1].
\end{equation}

We give a few examples below to illustrate this in $G(2,4)$:
\begin{center}
\begin{tabular}{c|c|cc}
Young diagram & $\lambda$ & $\alpha_1$ & $\alpha_2$ \\ \hline
$\emptyset$ & $[0,0]$ & $1$ & $2$ \\
$\tiny\yng(1)$ & $[1,0]$ & $1$ & $3$ \\
$\tiny\yng(2)$ & $[2,0]$ & $1$ & $4$ \\
$\tiny\yng(1,1)$ & $[1,1]$ & $2$ & $3$
\end{tabular}
\end{center}

Then, corresponding to the Young diagram $\lambda$, the Schubert cell
is represented by $k \times n$ matrix $\phi$ with the property that on
the $n$th row,
\begin{itemize}
\item the first $\alpha_n-1$ entries vanish,
\item the $\alpha_n$th entry equals $1$,
\item the entries above the $\alpha_n$th entry vanish.
\end{itemize}

The result has 
\begin{equation}
\sum_{i=1}^{k-1} (i) \left( \alpha_{i+1} - \alpha_i - 1 \right) \: + \:
(n - \alpha_k) k \: = \: k(n-k) \: - \: \sum_i \lambda_i 
\: = \: k (n-k) \: - \: | \lambda |
\end{equation}
undetermined entries, and hence describes a Schubert cycle of the same
dimension.  For example, in $G(2,4)$,
\begin{center}
\begin{tabular}{c|cc|c} 
Young diagram & $\alpha_1$ & $\alpha_2$ & dimension \\ \hline
$\emptyset$ & $1$ & $2$ & $4$ \\
$\tiny\yng(1)$ & $1$ & $3$ & $3$ \\
$\tiny\yng(2)$ & $1$ & $4$ & $2$ \\
$\tiny\yng(1,1)$ & $2$ & $3$ & $2$
\end{tabular}
\end{center}

Comparing to the description of \cite{gh}, we define a flag
\begin{equation}
V_1 \: = \: \langle e_1 \rangle, \: \: \:
V_2 \: = \: \langle e_1, e_2 \rangle, \: \: \:
V_n \: = \: \langle e_1, e_2, \cdots, e_n \rangle, \: \: \:
\cdots.
\end{equation}
Then, if we let $\Lambda_{\lambda}$ denote the closure of the Schubert
cell corresponding to $\lambda$, the description of \cite{Closset:2023bdr}
guarantees that 

**** FILL IN



****** NOTE:  Under dualizing $G(k,n)$ to $G(n-k,n)$,
take the transpose of the Young diagrams to map Schubert cycles between
them, ie, if $\lambda$ defines a Schubert cell in $G(k,n)$,
then $\lambda^T$ defines the same Schubert cell in $G(n-k,n)$.
******

***** Fulton Pra... ".. degeneracy loci.." may be a useful reference.




\subsection{Defect construction}

***** RELN to quiver given in Cyril et al eq (3.63) p 27

Given a Schubert cycle described as above, we can construct a quiver
gauge theory on a defect, following \cite[section 3.3.3]{Closset:2023bdr},
which we review here.
Write
\begin{equation}
\lambda \: = \: [ \lambda_1, \cdots, \lambda_{\ell}, 0, \cdots, 0 ].
\end{equation}
Then, the gauge group on the defect is
\begin{equation}
U(1) \times U(2) \times \cdots \times U(\ell).
\end{equation}
Furthermore, we partition $n$ into a $\ell+1$ blocks of dimensions
$M_i$ ($0 \leq i \leq \ell$), where
\begin{equation}
M_i \: = \: \left\{ \begin{array}{cl}
\lambda_i - \lambda_{i+1} + 1 & 1 \leq i < \ell, \\
\lambda_{\ell} - \ell + k & i = \ell, \\
n - \sum_{j=1}^{\ell} M_j = n-k+1-\lambda_1  & i = 0 .
\end{array} \right.
\end{equation}
For $1 \leq i \leq \ell$, there are Fermi multiplets in the
bifundamental of $U(i) \times \left( U(M_i) \subset U(n) \right)$
(the first factor local, the second factor global). 
In addition, there are bifundamentals $\varphi_i^{i+1}$ between
\begin{equation}
\left\{ \begin{array}{cl}
U(i) \times U(i+1) & i < \ell, \\
U(\ell) \times U(k) & i = \ell.
\end{array} \right.
\end{equation}


***** NOTE that we haven't used the $\alpha_i$ -- do they show up in
writing the superpotential?  If not, they can be eliminated from the discussion.
******

The reader should note that the defect gauge group and the $\varphi$
define a flag manifold $F(1,2,\cdots,\ell, k)$ over every point
of $G(k,n)$, following the GLSM description of flag manifolds 
in \cite{Donagi:2007hi} as, in this case, a $U(1) \times U(2)
\times \cdots \times U(\ell)$ gauge theory with bifundamentals,
in the same pattern as the $\varphi$.


*** This should tie into the description of Schubert cells in
Fulton's Intersection theroy (ch 14 or so).
******


Finally, there is a superpotential
\begin{equation}
\int d \theta \, \sum_{i=1}^{\ell} \varphi_i^{i+1} \varphi_{i+1}^{i+2}
\cdots \varphi_{\ell}^{\ell+1} \phi^{(i)} \Lambda^{(i)}.
\end{equation}

**** This seems to define the same map from the flag bundle above to 
the vector space $V$ of $G(k,V)$ that appears in
\cite[section 14.3]{fulton}, if we let $\sigma: S \hookrightarrow
V$ be inclusion in the notation of that reference.  Later in
\cite[chapter 14]{fulton} there's also a discussion of Schubert varieties...

**** Specifically, \cite[section 14.3]{fulton} describes Schubert varieties
in cohomology, and looks related.

**** See \cite[theorem 2.1]{buch1}.  There, given a space $X$ and a flag
$E_0 \rightarrow E_1 \rightarrow \cdots \rightarrow E_n$ of bundles
on $X$, they describe a construction of subspaces of $X$, which looks
similar.
They say that their quiver variety $\Omega_r$ is the scheme-theoretic
intersection of the zero sections of the bundle maps 
\begin{equation}
\wedge^{r_{ij}+1} E_i \: \longrightarrow \: \wedge^{r_{ij}+1} E_j,
\end{equation}
where the $r_{ij}$ are a set of nonnegative integers.
They also place a further requirement on the $r_{ij}$ -- can Schubert
cycles be described in this language?
********


We can summarize the mathematical interpretation of the defect
construction as follows.  Given $G(k,n)$, with universal subbundle $S$,
construct a flag bundle $\pi: F(1,2,\cdots,\ell,S) \rightarrow G(k,n)$
where $\ell$ is the number
of nonzero rows in the Young diagram $\lambda$.

*** If $\ell=k$, be careful -- Cyril et al's prescription gives a 
separate $U(k)$ on the defect.

Let 
\begin{equation}
S_{1} \: \stackrel{\varphi_1^2}{\hookrightarrow} \:
S_2 \: \stackrel{\varphi_2^3}{\hookrightarrow} \:
\cdots \:
\: \stackrel{\varphi_{\ell-1}^{\ell}}{\hookrightarrow} \:
S_{\ell} \: \stackrel{\varphi_{\ell}^{\ell+1}}{\hookrightarrow} \:
\pi^* S
\end{equation}
denote the universal subbundles $S_i$ (of rank $i$) 
on the flag bundle $F(1,2,\cdots,\ell,S)$
and their inclusions, ultimately into the pullback of the universal
subbundle $S$ on $G(k,n)$.

Partition the vector space $V = {\mathbb C}^n$ into $\ell+1$ distinct subspaces
$V_i$ ($0 \leq i \leq \ell$) of dimensions as follows:
\begin{equation}
\dim V_i \: = \: \left\{ \begin{array}{cl}
\lambda_i - \lambda_{i+1} + 1 & 1 \leq i < \ell, \\
\lambda_{\ell} - \ell + k & i = \ell, \\
n - k + 1 - \lambda_1 & i = 0,
\end{array} \right.,
\end{equation}
and define
\begin{equation}
F_i \: = \: V_0 \times \prod_{j \neq i} V_j \: \subset \: V,
\end{equation}
for $1 \leq i \leq \ell$.

{
    \color{blue} Let \(\lambda_0=n-k\) and \(\lambda_{\ell+1}=\ell-k+1\). Fix a basis \(\{e_i, 1\leq i\leq n\}\) for \(V=\mathbb{C}^n\) and let \(V_i=\langle e_{\lambda_0-\lambda_{i}+i+1},\dots,e_{\lambda_0-\lambda_{i+1}+i+1}\rangle\). Then we have \(\dim V_i=\lambda_i-\lambda_{i+1}+1\) for \(1\leq i\leq \ell\).
    Let \(\tilde{F}_j=\langle e_1,\dots,e_j\rangle\) for \(1\leq j\leq n\). Then they form a flag. Moreover, \(\tilde{F}_{n-k+i-\lambda_i}=\bigoplus_{j=0}^{i-1}V_i=\bigcap_{j=i}^\ell F_j\).
    % Let \(\tilde{F}_{n-k+i-\lambda_i}=F_i=\bigoplus_{j=0}^{i-1}V_i\), then \(\dim\tilde{F}_{n-k+i-\lambda_i}=n-k+i-\lambda_i\) and these vector spaces form a partial flag \(\tilde{F}_{n-k+1-\lambda_1}\subset\dots\subset\tilde{F}_{n-k+\ell-\lambda_\ell}\).
}

Let $Z_i$ denote the zero locus in $G(k,n)$ {\color{blue}$F(1,2,\cdots,\ell,S)$} of the map defined by
the composition
\begin{equation}
S_i \: \hookrightarrow \: S_{i+1} \hookrightarrow \: \cdots \:
S_{\ell} \: \hookrightarrow \: \pi^* S \: \hookrightarrow \:
V \: \longrightarrow \: V / F_i = V_i.
\end{equation}

{\color{blue}Let \(\Omega=\pi(Z_1 \cap Z_2 \cap \cdots \cap Z_{\ell})\) and \(\Sigma\) be a \(k\)-dimensional vector subspace of \(V\). Then \(\Sigma\in\Omega\) if and only if there exists a nested sequence of vector spaces \(\Sigma_1\subset\dots\Sigma_\ell\subseteq\Sigma\), where \(\Sigma_i\subseteq F_i\), and in particular, \(\Sigma_\ell\subseteq F_\ell=\tilde{F}_{n-k+\ell-\lambda_\ell}\), \(\Sigma_{\ell-1}\subseteq F_{\ell-1}\cap F_l=\tilde{F}_{n-k+\ell-1-\lambda_{\ell-1}},\dots,\Sigma_1\subseteq\bigcap_{j=1}^\ell F_j=\tilde{F}_{n-k+1-\lambda_1}\). This is if and only if \(\dim(\Sigma\cap \tilde{F}_{n-k+i-\lambda_i})\geq i\) for \(1\leq i\leq \ell\). Hence, \(\Omega\) is the Schubert variety corresponding to \(\lambda\).
% , more specifically, \(\Sigma_1\subseteq V_0=\tilde{F}_{n-k+1-\lambda_1}\)   
}

Then, the defect lives on the intersection
\begin{equation}     \label{eq:schubert-int}
Z_1 \cap Z_2 \cap \cdots \cap Z_{\ell}.
\end{equation}

**** Or a pushforward thereof, rather. ***



FILL IN

*********************************

Mathematically, let 
\begin{equation}
\tilde{F}_1 \subset \tilde{F}_2 \subset \cdots \subset \tilde{F}_{\ell} \subset
{\mathbb C}^n = V
\end{equation}
be a flag of subspaces, where
\begin{equation}
\dim \tilde{F}_i \: = \: n - k + i - \lambda_i,
\end{equation}
for $\lambda_i$ the number of boxes in row $i$,
then the Schubert cells is the intersection~(\ref{eq:schubert-int}) 
where the $Z_i$ are zero loci of maps defined by the composition
\begin{equation}
S_i \: \hookrightarrow \: S_{i+1} \hookrightarrow \: \cdots \:
S_{\ell} \: \hookrightarrow \: \pi^* S \: \hookrightarrow \:
V \: \longrightarrow \: V / \tilde{F}F_i.
\end{equation}

**** Note that the section above is on the flag bundle; presumably, we have
to push forward to the Grassmannian. ****

**** Note that in the physics construction we pick a basis for $V$.
Perhaps $F_1$ can be obtained from $\tilde{F}_1$ by taking the span of all
vectors not in $\tilde{F}_2$, and so forth for the remaining members.

**** Conversely, if we started with an analogous physics construction
that used $\tilde{F}_i$'s instead of Cyril et al's $F_i$'s, I'm wondering
if we could use some sort of linear algebra argument to demonstrate that
the resulting defects (defined by $\tilde{F}_i$) were equivalent to
those in Cyril et al's construction.

**** We spoke with Cyril on Nov 25 2023; he agreed with our interpretation
of his construction, and that it's not manifestly the same (but surely
equivalent) to the mathematics construction utilizing the flag
$\{ \tilde{F}_i \}$.


\subsection{Examples}


Next, we will describe the defects explicitly in some examples,
and also describe the sheaves described by the defects.

First consider ${\mathbb P}^{n-1} = G(1,n)$.
There, the Young tableau are single rows of boxes,
and so are completely determined by $\lambda_1$.
Let's assume $\lambda_1 \neq 0$, so $\ell = 1$.
Then, the gauge group on the defect is $U(1)$,
and
\begin{equation}
M_0 \: = \: n - \lambda_1, \: \: \: M_2 \: = \: \lambda_1.
\end{equation}
There is one bifundamental $\varphi_1^2$ in $U(1) \times U(1)$
(one local on defect, the other local in bulk),
and a bifundamental Fermi $\Lambda^{(1)}$ in $U(1) \times U(\lambda_1)$
(the first factor local on the defect, the second factor global, a subgroup of
$U(n)$), with superpotential
\begin{equation}
\int d\theta \, \varphi_1^2 \phi^{(1)} \Lambda^{(1)}.
\end{equation}
D terms imply that the $\varphi_1^2$ do not all vanish,
hence this will constrain $\phi^{(1)}$ to vanish.
Since $\phi^{(1)}$ is a $1 \times \lambda_1$ matrix,
this means that this defect lies along a codimension $\lambda_1$ locus,
which matches expectations for Schubert cells in projective spaces.
In fact, the Schubert variety in this case is ${\mathbb P}^{n-1-\lambda_1}$.

In passing, note that the number of $\Lambda^{(1)}$ matches the
codimension.

Now, let us interpret this mathematically.
From the bulk perspective, we interpret the $\Lambda^{(1)}$ as coupling to
${\cal O}^{\oplus \lambda_1}$ and $\varphi_1^2$ as coupling to ${\cal O}(-1)$,
so that the superpotential is realizing the cokernel of the map
\begin{equation}
{\cal O}(-1) \: \stackrel{\phi^{(1)}}{\longrightarrow} \:
{\cal O}^{\oplus \lambda_1},
\end{equation}
which can also be described as
\begin{equation}
{\cal O}(-1) \: \stackrel{\phi^{(1)}}{\longrightarrow} \:
{\cal O}^{n} \: \longrightarrow \: {\cal O}^{n - \lambda_1},
\end{equation}
where the second map is a projection.
The Schubert variety ${\mathbb P}^{n-1-\lambda_1}$ is the zero section 
of this map.

For $\lambda_1 = 1$, the cokernel of the map above
is precisely ${\cal O}_D$, the structure sheaf
of the divisor.  For $\lambda_1 > 1$, the cokernel, call it $C$, is
not\footnote{We would like to thank S.~Katz for explaining this to us.}
locally-free along the Schubert variety ${\mathbb P}^{n-1-\lambda_1}$.
If one projects ${\mathbb P}^{n-1} \rightarrow {\mathbb P}^{\lambda_1-1}$,
by mapping $x \mapsto (x_1, \cdots, x_{\lambda_1})$, the projection is not well-defined
along ${\mathbb P}^{n-1-\lambda_1}$, but is well-defined on the complement,
so we have a map
\begin{equation}
\pi: \: {\mathbb P}^{n-1} - {\mathbb P}^{n-1-\lambda_1} \: \longrightarrow
\: {\mathbb P}^{\lambda_1-1}.
\end{equation}
On the complement, the cokernel is isomorphic to $\pi^* T {\mathbb P}^{\lambda_1-1}(1)$,
locally-free of rank $\lambda_1-1$.  Along ${\mathbb P}^{n-1-\lambda_1}$, the
map ${\cal O}(-1) \rightarrow {\cal O}^{\lambda_1}$ vanishes, so $C$ requires
$k$ local generators at any point of ${\mathbb P}^{n-1-\lambda_1}$, and so is
not locally-free of rank $\lambda_1-1$ there.

To put this in context, the normal bundle to the Schubert variety
${\mathbb P}^{n-1-\lambda_1}$ in ${\mathbb P}^{n-1}$ is
${\cal O}(1)^{\lambda_1}$, since ${\mathbb P}^{n-1-\lambda_1}$ is a complete
intersection of $\lambda_1$ sections of ${\cal O}(1)$.

As a result, we cannot interpret this sheaf in terms of the normal bundle
or tangent bundle or structure sheaf to the Schubert variety,
except in the special case $\lambda_1 = 1$.

We have not yet taken into account the gauge theory along the defect.

**** From the defect $U(1)$ perspective, this defines something like
\begin{equation}
{\cal O}(1) \: \stackrel{\phi^{(1)}}{\longrightarrow} \:
{\cal O}(1)^{\oplus \lambda_1}
\end{equation}
Helpful?


**** Recall that for $S \subset X$,
\begin{equation}
0 \: \longrightarrow TS \: \hookrightarrow \: TX|_S \: \longrightarrow \:
N_{S/X} \: \longrightarrow \: 0.
\end{equation}

*******************

Next, consider $G(k,n)$, and the case that
$\lambda = \tiny\yng(1)$.
Here, $\ell = 1$, $M_0 = n-k$, $M_1 = k$.  The defect gauge group is $U(1)$,
there is a bifundamental chiral $\varphi_1^2$ in $U(1) \times U(k)$
(both gauged), and a bifundamental Fermi $\Lambda^{(1)}$ in
$U(1) \times U(k)$ (the first factor gauged, the second factor global),
with superpotential
\begin{equation}
\int d\theta \, \varphi_1^2 \phi^{(1)} \Lambda^{(1)}
\end{equation}

From the bulk, the complex complex for $\tiny\yng(1)$ above, on $G(k,n)$, is
\begin{equation}
S \: \stackrel{\phi^{(1)}}{\longrightarrow} \: {\cal O}^{k}
\end{equation}
(since here $M_1 = k$, $M_0 = n-k$).
The cokernel of this complex is\footnote{
We would like to thank S.~Katz for an explanation.
}
a sheaf supported on a Schubert cycle.
On its support, it is generically a line bundle, but it degenerates in
higher codimension.  He says its Chern character is computed as
\begin{equation}
{\rm ch} \: = \: k - {\rm ch}(S) \: = \: k - (k -  \sigma_1 + \cdots)
\: = \: \sigma_1 + \cdots,
\end{equation}
consistent with being supported on a Schubert cycle.

Now, in $G(k,n)$, the normal bundle to the Schubert variety corresponding to
$\tiny\yng(1)$ is the determinant of the restriction of the universal
subbundle $S$ **** OR ITS INVERSE *****.
Clearly, $\Lambda^{(1)}$ does not itself couple to the normal bundle.

Further, the cokernel of the sequence above also 
can not be the normal
bundle, as it has the wrong rank. 
We can describe this more explicitly as follows.
First, write the sequence as
\begin{equation}
S \: \longrightarrow \: {\cal O}^n \: \longrightarrow \: {\cal O}^{n}/{\cal O}^{n-k},
\end{equation}
where the first map is inclusion, and the second is a choice of projection
of ${\mathbb C}^n$ onto ${\mathbb C}^k$.  The kernel of the projection
${\mathbb C}^n \rightarrow {\mathbb C}^k$ is a $(n-k)$-dimensional subspace
$V$.  The cokernel, call it $C$, is supported on the Schubert cycle of
$k$-dimensional subspaces $W \subset {\mathbb C}^n$ such that
$W \cap V \neq 0$.  If $W$ is such that 
\begin{equation}
\dim(W \cap V) \: = \: r,
\end{equation}
then the sheaf $C$ has $r$ generators locally at $W$ in $G(k,n)$,
and $r \geq 1$ for $W$ in the Schubert cycle.  Generically, $r=1$ so that
$C$ is a line bundle on a dense open subset of the Schubert cycle.
However,
the subset where $r \geq 2$ is nonempty (except for small values of $k$ or
$n-k$), so that $C$ is not a line bundle, and so cannot be a normal
bundle or dual of a normal bundle.

**** Leonardo notes that this arises in degeneracy loci constructions.



*** From the defect $U(1)$ perspective, this is something like
\begin{equation}
{\cal O}(1)^k \: \stackrel{\phi^{(1)}}{\longrightarrow} \:
{\cal O}(1)^k
\end{equation}
Helpful?

*** If we let ${\cal L}$ denote the defect $U(1)$ line, then altogether,
combining bulk and defect,
this is something like
\begin{equation}
S \otimes {\cal L} \: \stackrel{\phi^{(1)}}{\longrightarrow} \:
{\cal L}^k
\end{equation}
Helpful?

***** Here's what I'm taking from Sheldon's message:  these sheaves may have
support on the Schubert cycle, but aren't otherwise obvious choices such as
the tangent sheaf or normal sheaf or some such.  That's a pity.
Can anything noteworthy be said about them?  Also, can we check in more
general cases that they're still supported on Schubert cycles?

***** In case it's helpful, the cotangent complex of the derived zero
locus is described in
\cite[section 4.1]{Sharpe:2019yag}.
However, its cotangent complex is different from the sheaf
we're seeing here.
*****


Another example.  Consider again $G(k,n)$, and the Schubert cycle corresponding
to
\begin{equation}
\lambda \: = \: \tiny\yng(1,1).
\end{equation}
Here, $\ell=2$, the defect gauge group is $U(1) \times U(2)$,
\begin{equation}
M_1 \: = \: 1, \: \: \:
M_2 \: = \: k-1, \: \: \:
M_0 \: = \: n-k.
\end{equation}
Along the defect there are chirals $\varphi_1^2$ in the bifundamental
of $U(1) \times U(2)$ (both gauged) and $\varphi_2^3$ in the
bifundamental of $U(2) \times U(k)$ (both gauged, the first along the
defect, the second in bulk), as well as Fermi fields
$\Lambda^{(1)}$ in the bifundamental of $U(1) \times U(M_1)$
(the first local on the defect, the second global in bulk)
and $\Lambda^{(2)}$ in the bifundamental of $U(2) \times U(M_2)$
(the first local on the defect, the second global in bulk),
and a superpotential
\begin{equation}
\int d \theta \left( \varphi_1^2 \varphi_2^3 \phi^{(1)} \Lambda^{(2)} \: + \:
\varphi_2^3 \phi^{(2)} \Lambda^{(2)} \right),
\end{equation}
which enforce (via $F$-terms) the constraints
\begin{equation}
\varphi_1^2 \varphi_2^3 \phi^{(1)} \: = \: 0, \: \: \:
\varphi_2^3 \phi^{(2)} \: = \: 0.
\end{equation}






\section{Symplectic Grassmannians}

According to Leonardo, there is *not* a simple relationship between
Schubert cycles in ordinary Grassmannians
and Schubert cycles in sympletic, orthogonal Grassmannians.
Pity -- I was thinking one could repeat Cyril's construction in the
symplectic and orthogonal cases, but it sounds like something more
is required.


****** \cite{bkt} has some basic info on Schubert cycles in symplectic
Grassmannians.  Idea is that instead of looking at flags,
look at isotropic flags, satisfying some rank conditions.
*****

**** In particular, for symplectic Grassmannians, from 
\cite[section 1.1]{bkt}, the cohomology is generated over the
integers by a subset of all Young diagrams, namely those for which
$\lambda_1 > \lambda_2 > \cdots$.  So for example, for $LG(2,4)$,
the allowed Young diagrams are
\begin{equation}
1, \: \: \:
\yng(1), \: \: \:
\yng(2), \: \: \:
\yng(2,1),
\end{equation}
but not
\begin{equation}
\yng(1,1), \: \: \:
\yng(2,2).
\end{equation}
Leonardo has confirmed this.

**** Also, in case it's helpful, Lagrangian Grassmannians are zero sections
of bundles over $G(n,2n)$. *****


Recall from \cite{Gu:2020oeb} that the GLSM for the symplectic
Grassmannian $SG(k,2n)$ is a $U(k)$ gauge theory with 
$2n$ chirals $\Phi^a_{\pm i}$ in the fundamental representation $V$,
and one chiral superfield $q_{ab}$ in the representation $\wedge^2 V^*$,
with superpotential
\begin{equation}
W \: = \: \sum_{i=1}^n q_{ab} \Phi^a_{+i} \Phi^b_{-i}.
\end{equation}
The superpotential enforces the isotropy condition
\begin{equation}
\sum_{i=1}^n \phi^a_{+i} \phi^b_{-i} \: = \:
\sum_{i=1}^n \phi^b_{+i} \phi^a_{-i}.
\end{equation}

Now, if we give a mass to a subset, we need to do so in such a way
as to preserve the isotropy condition.

Here's a guess for the first nontrivial Schubert cycle, corresponding
to a single box:
break $\phi_{\pm}$ (each a $k \times n$ matrix)
into a $k \times k$ matrix $\phi_{\pm}^{(1)}$ and a 
$k \times (n-k)$ matrix $\phi_{\pm}^{(0)}$.
Let $\varphi_{1 \pm}^2$ be a pair of bifundamentals $U(1) \rightarrow U(k)$,
and $\Lambda_{\pm}^{(1)}$ be two sets of Fermis in the fundamental,
and define
\begin{equation}
W \: = \: \varphi_{1+}^2 \phi^{(1)}_+ \Lambda^{(1)}_+ \: + \:
\varphi_{1-}^2 \phi^{(1)}_- \Lambda^{(1)}_-.
\end{equation}
That will give a mass to a $1\times k$ submatrix of both
$\phi_{\pm}$, which at least potentially is consistent with the isotropy
condition.

**** What constraints exist on Young diagrams describing Schubert cells
in symplectic Grassmannians?

Let's rewrite to take into account the global $SU(2)$ symmetry, for
which $\pm$ is a doublet.  Now, we have 3 factors, but, 
\begin{equation}
2 \otimes 2 \otimes 2 \: = \: 4 \oplus 2 \oplus 2
\end{equation}
contains no singlets, so to rewrite in a manifestly $SU(2)$-invariant
form, we need to group the 3 factors into a product of two.  We can write
either
\begin{equation}
\omega^{ab} \varphi_{1 a}^2 \phi_a^{(1)} \Lambda_b^{(1)}
\end{equation}
or
\begin{equation}
\omega^{ab} \varphi_{1 a}^2 \phi^{(1)}_b \Lambda_b^{(1)}
\end{equation}
or
\begin{equation}
\omega^{ab} \varphi_{1 b}^2 \phi^{(1)}_a \Lambda_b^{(1)}
\end{equation}
All three are consistent with $SU(2)$.  Let's work out more examples of
Schubert cycles, that should make it clear which of these is correct.

**** Alternate:

Alternatively, maybe there's only one $\varphi_1^2$, not two.
In that case, the defect superpotential is just
\begin{equation}
\omega^{ab} \varphi_1^2 \phi^{(1)}_a \Lambda^{(1)}_b
\end{equation}

Note that this expression immediately generalizes to other quivers of
\cite{Closset:2023bdr}:  we use the same quiver and same $\varphi$,
and just constract everything else along global $SU(2)$ indices.

One consistency check:  are we giving a mass to all of the $\Lambda$'s?
We need to lift the same number of bosonic modes.  That may help cut down
on the possibilities.




TO DO:
\begin{itemize}
\item Generalize to other Young diagrams.  For the last proposal above,
that's easy.  If we need a different proposal, we'll have to figure out the
generalization.
%
\item Compare to known Schubert cycles -- presumably results exist for
$LG(2,4)$, for example.
%
\item Compute Grothendieck polynomials to check against known results.
\end{itemize}


\appendix


\section{Comparison of Schubert cells in $G(3,6)$ and $LG(3,6)$}

**** Briefly, the restriction of Schubert cells in $G(3,6)$ are *not*
the Schubert cells in $LG(3,6)$.  Below is a message from Leonardo, Nov 16:

As far as I can see, if one considers the inclusion 
$LG(3,6) \rightarrow  G(3,6)$
and pulls back cohomology classes of then
\begin{eqnarray}
i^*(1) & = & (1)
\\
i^*(2) & = & (2)
\\
i^*(2,1) & = & (2,1) + (3)
\\
i^*(3) & = & (3).
\end{eqnarray}
So one obtains more than 1 Schubert class, even if one pull back classes for strict partitions.

This is consistent to the fact that
\begin{equation}
i^*((2,11) +(3)) = i^*((2) \cdot (1)) = i^*(2) \cdot i^*(1) = (2) * (1) 
= 2 (3) + (2,1).
\end{equation}

\section{Example: Schubert cells in $LG(2,4)$}

These are computed with respect to the symplectic form
\begin{equation}
\left[ \begin{array}{cc}
0 & +I \\ -I & 0 \end{array} \right].
\end{equation}
where $I$ denotes the $2 \times 2$ identity matrix.

We need to find a set of subspaces $F_i$ ($i \in \{1, \cdots, 2n\}$ for
$SG(k,2n)$), where the dimension of $F_i$ is $i$,
such that $F_{n+i} = F_{n-i}^{\perp}$.
Here, we can take
\begin{eqnarray}
F_1 & = & \langle e_1 \rangle,
\\
F_2 & = & \langle e_1, e_2 \rangle \: = \: F_2^{\perp},
\\
F_3 & = & \langle e_1, e_2, e_4 \rangle \: = \: F_1^{\perp},
\\
F_4 & = & {\mathbb C}^4.
\end{eqnarray}

**** BELOW IS NONSENSE:

Below are some examples of Schubert cells in $LG(2,4)$:
\begin{eqnarray}
\emptyset: & & \phi = \left[ \begin{array}{cccc}1 & 0 & a & b \\
0 & 1 & c & d \end{array} \right], \: \: \: b=c, \\
& & 
\phi_+ = \left[ \begin{array}{cc} 1 & 0 \\ 0 & 1 \end{array} \right],
\: \: \:
\phi_- = \left[ \begin{array}{cc} a & b \\ b & d \end{array} \right];
\\
\tiny\yng(1): & & \phi = \left[ \begin{array}{cccc} 1 & c & 0 & b \\
0 & 0 & 1 & a \end{array} \right], \: \: \: 1 + ac = 0;
\\
\tiny\yng(2): & & \phi = \left[ \begin{array}{cccc} 1 & 0 & b & 0 \\
0 & 0 & 0 & 1 \end{array} \right];
\\
\tiny\yng(2,1): & & \phi = \left[ \begin{array}{cccc} 0 & 0 & 1 & 0 \\
0 & 0 & 0 & 1 \end{array} \right].
\end{eqnarray}
Briefly, these are obtained by starting with the Schubert cells used
in \cite{Closset:2023bdr} for $G(2,4)$, and then imposing the isotropy
condition, with the exception of the cell for $\tiny\yng(2,1)$.

*** Should I be worried about the quadratic condition on the coefficients
in $\tiny\yng(1)$ ???


\begin{thebibliography}{199}

\addcontentsline{toc}{section}{References}


\bibitem{Closset:2023bdr}
C.~Closset and O.~Khlaif,
``Grothendieck lines in 3d $\mathcal{N}=2$ SQCD and the quantum K-theory of the Grassmannian,''
{\tt arXiv:2309.06980 [hep-th]}.

\bibitem{Gu:2020oeb}
W.~Gu, E.~Sharpe and H.~Zou,
``GLSMs for exotic Grassmannians,''
JHEP \textbf{10} (2020) 200,
{\tt arXiv:2008.02281 [hep-th]}.

\bibitem{fp} W. Fulton, P. Pragacz, {\it Schubert varieties and degeneracy
loci}, Lecture notes in math. 1689, Springer, Berlin, 1998.

\bibitem{bkt} A. Buch, A. Kresch, H. Tamvakis, ``Quantum Pieri rules for
isotropic Grassmannians,''
available at
{\tt https://sites.math.rutgers.edu/~asbuch/papers/qprig.pdf}


\bibitem{gh} P. Griffiths, J. Harris, {\it Principles of algebraic
geometry}, John Wiley \& Sons, New York, 1978.

\bibitem{Sharpe:2019yag}
E.~Sharpe,
``Categorical equivalence and the renormalization group,''
Fortsch. Phys. \textbf{67} (2019)  1910019,
{\tt arXiv:1903.02880 [hep-th]}.

\bibitem{Donagi:2007hi}
R.~Donagi and E.~Sharpe,
``GLSM's for partial flag manifolds,''
J. Geom. Phys. \textbf{58} (2008) 1662-1692,
{\tt arXiv:0704.1761 [hep-th]}.

\bibitem{fulton} W. Fulton, {\it Intersection theory}, second edition,
Springer-Verlag, Berlin, 1998.

\bibitem{buch1} A. Buch, ``Grothendieck classes of quiver varieties,''
Duke Math. J. {\bf 115} (2002) 75-103,
{\tt arXiv:math/0104029 [math.AG]}.


\end{thebibliography}

\end{document}

